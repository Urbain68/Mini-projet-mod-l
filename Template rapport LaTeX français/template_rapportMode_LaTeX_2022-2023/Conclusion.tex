\chapter{Conclusion}
\label{ch:Conclusion} 

\section{Conclusion sur les modèles proposés}
La conclusion résume les grandes lignes du rapport (principe des solutions, principaux résultats et limitations). Elle doit contenir le message essentiel que l’on souhaite faire passer et montrer l’intérêt des approches proposées dans le rapport.

\section{Perspectives}

Plusieurs pistes sont envisageables afin de complexifier le modèle et de le rendre plus en accord avec la réalité.

On peut par exemple remplacer les constantes positives dans (\ref{eq:syst_continu}) par des fonctions continues afin de modéliser des changements environnementaux, comme la succession des saisons ou encore des périodes de sécheresse. On pourrait aussi ajouter des termes de migration au système.

Une autre piste serait d'ajouter d'autres espèces, ce qui pourrait introduire une compétition entre deux espèces de prédateurs pour une même proie ou bien permettrait de modéliser la prolifération des espèces végétales qui servent de nourriture aux proies afin de représenter une chaîne alimentaire plus complexe.

\vspace{0.5cm}
Enfin, afin de rendre les simulation plus réalistes, il pourrait être avantageux de faire des mesures dans un milieu naturel afin d'obtenir une estimation de la valeur que l'on pourrait assigner aux différents coefficients $\alpha, \beta, \gamma, \rho, \sigma$ et $\nu$.
