\chapter*{Annexe 1 (si besoin)}
\addcontentsline{toc}{chapter}{Annexe 1}

%changer le format des sections, subsections pour apparaittre sans le num de chapitre
\makeatletter
\renewcommand{\thesection}{\@arabic\c@section}
\makeatother

%recommencer la numérotation des section à "1"
\setcounter{section}{0}

\section*{Style de rédaction}
Un rapport technique se rédige dans un style neutre. Il convient en particulier d’éviter un vocabulaire familier ou trop littéraire. On recherche la concision et la clarté en évitant les phrases trop longues et les termes imprécis.

L’usage des pronoms personnels (par exemple \textit{nous}) est à limiter à de rare cas précis (par exemple pour mettre l’accent sur un choix déterminant ou pour se démarquer par rapport aux autres).

Une énumération est constituée d’objets qui se situent sur le même plan, par ailleurs les éléments de l’énumération doivent avoir la même nature grammaticale.

Il est fondamental de relire le rapport avec un regard extérieur, en particulier pour vérifier le sens des phrases, ainsi que leur clarté.
