\chapter{Modèle à état discret (\textit{2 pages max})}
\label{ch:modele_discret}

Ce chapitre se propose d'étudier le système présenté en (\ref{eq:syst_continu}) en le transformant sous la forme d'un système à temps discret, puis en en étudiant les points d'équilibre et leur stabilité. Enfin, les hypothèses sous-jacentes du modèle proposé seront étudiées plus en détail dans \ref{sec:Hypotheses}

\section{Modèle proposé}

Le système continu d'origine du modèle étudié ici est le suivant :
\begin{equation}
    \label{eq:syst_continu}
    \begin{cases}
        \dot{x}(t) = \alpha x(t) - \beta x(t) y(t) - \gamma x(t)^2 \\
        \dot{y}(t) = - \rho y(t) + \sigma x(t) y(t) - \nu y(t)^2
    \end{cases}
\end{equation}
où $\alpha$, $\beta$, $\gamma$, $\rho$, $\sigma$ et $\nu$ sont des constantes positives.

Pour un système réel, $\alpha$ représente le taux de naissance des proies dans l'hypothèse où elles ont accès à autant de ressources que nécessaires.
Le facteur $\beta$ traduit la probabilité qu'une proie se fasse dévorer par un prédateur et le facteur $\sigma$ traduit cette même interaction du point de vue du prédateur, tout en prenant en compte la valuer nutritive de la proie. En effet, dans des milieux réels, on observe qu'un prédateur doit se nourrir de plusieurs proies afin de se reproduire avec succès.
Le facteur $\rho$ représente quant à lui le taux de décès des prédateurs. En effet, on considère que les prédateurs se nourrissent exclusivement des proies de l'espèce $x$. En l'absence de proies, la population de prédateurs diminue donc progressivement proportionnellement au facteur $\rho$.
Enfin, $\gamma$ et $\nu$ modélisent la compétition qui a lieu à l'intérieur d'une espèce pour s'approprier les ressources nécessaires à la croissance.

Afin de pouvoir mener une étude plus approfondie de ce système, on cherche ensuite à le discrétiser dans le temps sous la forme de deux suites chaînées définies par récurrence.

En utilisant l'approximation $\dot{x}(t) = x_{n+1} - x_n$ (resp. $\dot{y}(t) = y_{n+1} - y_n$), on obtient immédiatement le système à temps discret suivant :
\begin{equation}
    \label{eq:syst_discret}
    \begin{cases}
        x_{n+1} = (1 + \alpha) x_n - \beta x_n y_n - \gamma x_n^2 \\
        y_{n+1} = (1 - \rho) y_n + \sigma x_n y_n - \nu y_n^2
    \end{cases}
\end{equation}


\section{Etude théorique des points d'équilibre}

Maintenant que nous disposons d'un système d'équations modélisant la dynamique de deux populations de proies et de prédateurs, il s'agit d'en étudier les points d'équilibre.

On peut tout d'abord remarquer que le point $(x_n, y_n) = (0, 0)$ est un point d'équilibre trivial assez peu intéressant pour le système \ref{eq:syst_discret}.

Si l'on cherche ensuite d'autres potentiels points d'équilibre, on est amené à résoudre le système d'équations :
\begin{equation}
    \begin{cases}
        x^* = (1 + \alpha) x^* - \beta x^* y^* - \gamma (x^*)^2 \\
        y^* = (1 - \rho) y^* + \sigma x^* y^* - \nu (y^*)^2
    \end{cases}
\end{equation}
d'inconnues $x^*$ et $y^*$ supposées positifs et non nuls.

Il reste alors trois cas à traiter : $x^* = 0$ et $y^* > 0$ ; $x^* > 0$ et $y^* = 0$ et enfin $x^* > 0$ et $y^* > 0$.
Pour les deux premiers cas, on obtient après résolution les points d'équilibre suivant : $(0, - \frac{\nu}{\rho})$ qui n'est pas cohérent car on doit avoir $y^* > 0$ et $(\frac{\alpha}{\gamma}, 0)$, que l'on va noter par la suite $(\bar{x}, 0)$.

Sans se soucier pour l'instant des conditions d'existence, en simplifiant les deux lignes par $x^*$ (resp. $y^*$) puis en résolvant, on obtient une solution possible :
\begin{equation}
    \label{eq:equilibre}
    \begin{cases}
        x^* = \frac{1}{\gamma} (\alpha - \beta y^*) \\
        y^* = \frac{\sigma \alpha - \rho \gamma}{\beta + \nu \gamma}
    \end{cases}
\end{equation}

Or pour qu'un tel point d'équilibre existe, il doit vérifier $x^* > 0$ et $y^* > 0$, c'est-à-dire les deux conditions : $\sigma \alpha > \rho \gamma$ et $\alpha > \beta y^*$.
\vspace{5pt}

On s'intéresse maintenant à la stabilité des deux points d'équilibre ainsi obtenus.

Pour cela, on va chercher à calculer le linéarisé tangent du système \ref{eq:syst_discret} aux points d'équilibre $(0, 0)$ et \ref{eq:equilibre}.
Avant cela, on va poser les notations suivantes :
\begin{align}
    f : (x, y) \rightarrow (1 + \alpha) x - \beta x y - \gamma x^2 \\
    g : (x, y) \rightarrow (1 - \rho) y + \sigma x y - \nu y^2
\end{align}
de sorte qu'on a : $x_{n+1} = f(x_n, y_n)$ et $y_{n+1} = g(x_n, y_n)$.
\newline
Alors on a les dérivées partielles suivantes :
\begin{align}
    \frac{\partial f}{\partial x}(x,y) &= 1 + \alpha - \beta y - 2 \gamma x \\
    \frac{\partial f}{\partial y}(x,y) &= - \beta x \\
    \frac{\partial g}{\partial x}(x,y) &= \sigma y \\
    \frac{\partial g}{\partial y}(x,y) &= 1 - \rho + \sigma x - 2 \nu y 
\end{align}

D'où les matrices suivantes pour le linéarisé tangent aux trois points d'équilibre :
\begin{equation}
    J_{1}(0,0) = 
  \begin{bmatrix}
    1 + \alpha & 0 \\
    0 & 1 - \rho
  \end{bmatrix}
  \hspace{5pt}
  \text{et}
  \hspace{5pt}
  J_{2}(x^*, y^*) = 
  \begin{bmatrix}
    1 - \alpha + \beta y^* & - \beta x^* \\
    \sigma y^* & 1 - \rho + \sigma x^* + 2 \nu y^*
  \end{bmatrix}
  \hspace{5pt}
  \text{et}
  \hspace{5pt}
  J_{3}(\bar{x}, 0) = 
  \begin{bmatrix}
    1 - \alpha & - \beta \bar{x} \\
    0 & 1 - \rho + \sigma \bar{x}
  \end{bmatrix}
\end{equation}

On peut alors calculer l'expression du polynôme caractéristique de ces deux matrices et en déduire par le critère de Jury une condition de stabilité pour ces points d'équilibre.

Pour l'équilibre en $(0,0)$, on trouve : $\chi_{(0,0)} (X) = (X - (1 + \alpha))(X - (1 - \rho))$ et on remarque que $(1 + \alpha)$ est une valeur propre de module strictement supérieur à $1$. Ainsi, le point $(0, 0)$ est un point d'équilibre instable pour le système considéré.

Pour ce qui est de l'équilibre en $(x^*, y^*)$, on peut utiliser le critère de Jury, présenté dans \cite{Chevet22fr} au chapitre \textbf{4.4.2.3.2}, pour obtenir une série d'inégalités entre les différents paramètres du problème qui caractérisent exactement les conditions d'équilibre stable.

\section{Hypothèses du modèle}
\label{sec:Hypotheses}

Le modèle proposé plus haut se base implicitement sur plusieurs hypothèses qui simplifient considérablement l'évolution d'une population.

En premier lieux, il suppose que l'évolution des population de proies et de prédateurs ne dépende que des conditions initiales imposées au système. En particulier, ce modèle considère que tous les facteurs multiplicatifs présents dans le système (\ref{eq:syst_discret}) sont des constantes. Ainsi, on ne prend pas en compte l'influence de facteurs environnementaux comme les saisons ou les évènements météorologiques extraordinaires.

De plus, le modèle de compétition intra-espèce suppose aussi que les deux espèces sont parfaitement isolées du reste de l'univers. Par exemple, on ne suppose pas la présence d'une deuxième espèce de prédateurs qui viendrait réduire la population de proies.
\vspace{15pt}


En début de chapitre, il est conseillé d'annoncer ce qui va être traité par la suite et dans quel but. Ensuite, il s’agit de traiter la problématique exposée en début du chapitre. Il est important de citer les différentes sources utilisées pour réaliser votre mini-projet.

Il s'agit ensuite de réexposer le problème de modélisation en s’appuyant sur un formalisme mathématique détaillé, de préciser et d'analyser les hypothèses de modélisation, etc.

Ce chapitre permettra de répondre notamment aux questions I.1, I.2 et I.3 du sujet, en précisant les méthodes théoriques utilisées.