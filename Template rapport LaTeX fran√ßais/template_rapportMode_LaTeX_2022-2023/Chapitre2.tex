\chapter{Modèle à état discret (\textit{2 pages max})}
\label{ch:modele_discret}

Ce chapitre se propose d'étudier le système présenté en (\ref{eq:syst_continu}) en le transformant sous la forme d'un système à temps discret, puis en en étudiant les points d'équilibre et leur stabilité.

\section{Modèle proposé}

Le système continu d'origine du modèle étudié ici est le suivant :
\begin{equation}
    \label{eq:syst_continu}
    \begin{cases}
        \dot{x}(t) = \alpha x(t) - \beta x(t) y(t) - \gamma x(t)^2 \\
        \dot{y}(t) = - \rho y(t) + \sigma x(t) y(t) - \nu y(t)^2
    \end{cases}
\end{equation}
où $\alpha$, $\beta$, $\gamma$, $\rho$, $\sigma$ et $\nu$ sont des constantes positives.

Afin de pouvoir mener une étude plus approfondie de ce système, on cherche ensuite à le discrétiser dans le temps sous la forme de deux suites chaînées définies par récurrence.

En utilisant l'approximation $\dot{x}(t) = x_{n+1} - x_n$ (resp. $\dot{y}(t) = y_{n+1} - y_n$), on obtient immédiatement le système à temps discret suivant :
\begin{equation}
    \label{eq:syst_discret}
    \begin{cases}
        x_{n+1} = (1 + \alpha) x_n - \beta x_n y_n - \gamma x_n^2 \\
        y_{n+1} = (1 - \rho) y_n + \sigma x_n y_n - \nu y_n^2
    \end{cases}
\end{equation}


\section{Etude théorique des points d'équilibre}

Maintenant que nous disposons d'un système d'équations modélisant la dynamique de deux populations de proies et de prédateurs, il s'agit d'en étudier les points d'équilibre.

On peut tout d'abord remarquer que le point $(x_n, y_n) = (0, 0)$ est un point d'équilibre trivial assez peu intéressant pour le système \ref{eq:syst_discret}.

Si l'on cherche ensuite d'autres potentiels points d'équilibre, on est amené à résoudre le système d'équations :
\begin{equation}
    \begin{cases}
        x^* = (1 + \alpha) x^* - \beta x^* y^* - \gamma (x^*)^2 \\
        y^* = (1 - \rho) y^* + \sigma x^* y^* - \nu (y^*)^2
    \end{cases}
\end{equation}
d'inconnues $x^*$ et $y^*$ supposées positifs et non nuls.

Sans se soucier pour l'instant des conditions d'existence, en simplifiant les deux lignes par $x^*$ (resp. $y^*$) puis en résolvant, on obtient une solution possible :
\begin{equation}
    \label{eq:equilibre}
    \begin{cases}
        x^* = \frac{1}{\gamma} (\alpha - \beta y^*) \\
        y^* = \frac{\sigma \alpha - \rho \gamma}{\beta + \nu \gamma}
    \end{cases}
\end{equation}

Or pour qu'un tel point d'équilibre existe, il doit vérifier $x^* > 0$ et $y^* > 0$, c'est-à-dire les deux conditions : $\sigma \alpha > \rho \gamma$ et $\alpha > \beta y^*$.
\newline

On s'intéresse maintenant à la stabilité des deux points d'équilibre ainsi obtenus.

Pour cela, on va chercher à calculer le linéarisé tangent du système \ref{eq:syst_discret} aux points d'équilibre $(0, 0)$ et \ref{eq:equilibre}.
Avant cela, on va poser les notations suivantes :
\begin{align}
    f : (x, y) \rightarrow (1 + \alpha) x - \beta x y - \gamma x^2 \\
    g : (x, y) \rightarrow (1 - \rho) y + \sigma x y - \nu y^2
\end{align}
de sorte qu'on a : $x_{n+1} = f(x_n, y_n)$ et $y_{n+1} = g(x_n, y_n)$.
\newline
Alors on a les dérivées partielles suivantes :
\begin{align}
    \frac{\partial f}{\partial x}(x,y) &= 1 + \alpha - \beta y - 2 \gamma x \\
    \frac{\partial f}{\partial y}(x,y) &= - \beta x \\
    \frac{\partial g}{\partial x}(x,y) &= \sigma y \\
    \frac{\partial g}{\partial y}(x,y) &= 1 - \rho + \sigma x - 2 \nu y 
\end{align}

D'où les matrices linéarisées tangentes suivantes :
\[
  J_{1}(0,0) = 
  \begin{bmatrix}
    1 + \alpha & 0 \\
    0 & 1 - \rho
  \end{bmatrix}
\]

\[
  J_{2}(x^*, y^*) = 
  \begin{bmatrix}
    1 - \alpha + \beta y^* & - \beta x^* \\
    \sigma y^* & 1 - \rho + \sigma x^* + 2 \nu y^*
  \end{bmatrix}
\]

On peut alors calculer l'expression du polynôme caractéristique de ces deux matrices et en déduire par le critère de Jury une condition de stabilité pour ces points d'équilibre.

Pour l'équilibre $(0,0)$, on trouve : $\chi_{(0,0)} (X) = (X - (1 + \alpha))(X - (1 - \rho))$ et on remarque que $(1 + \alpha)$ est une valeur propre de module strictement supérieur à $1$. Ainsi, le point $(0, 0)$ est un point d'équilibre instable pour le système considéré.

Pour ce qui est de l'équilibre $(x^*, y^*)$, on peut utiliser le critère de Jury pour obtenir une série d'inégalités entre les différents paramètres du problème.

\section{Hypothèses du modèle}


En début de chapitre, il est conseillé d'annoncer ce qui va être traité par la suite et dans quel but. Ensuite, il s’agit de traiter la problématique exposée en début du chapitre. Il est important de citer les différentes sources utilisées pour réaliser votre mini-projet.

Il s'agit ensuite de réexposer le problème de modélisation en s’appuyant sur un formalisme mathématique détaillé, de préciser et d'analyser les hypothèses de modélisation, etc.

Ce chapitre permettra de répondre notamment aux questions I.1, I.2 et I.3 du sujet, en précisant les méthodes théoriques utilisées.