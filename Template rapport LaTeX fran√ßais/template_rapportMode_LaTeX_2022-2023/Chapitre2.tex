\chapter{Modèle à état continu (\textit{2 pages max})}
\label{ch:modele_continu}

En début de chapitre, il est conseillé d'annoncer ce qui va être traité par la suite et dans quel but. Ensuite, il s’agit de traiter la problématique exposée en début du chapitre. Il est important de citer les différentes sources utilisées pour réaliser votre mini-projet.

Il s’agit ensuite de réexposer le problème de modélisation en s’appuyant sur un formalisme mathématique détaillé, de préciser et d'analyser les hypothèses de modélisation, etc.

Ce chapitre permettra de répondre notamment aux questions I.1, I.2 et I.3 du sujet, en précisant les méthodes théoriques utilisées.