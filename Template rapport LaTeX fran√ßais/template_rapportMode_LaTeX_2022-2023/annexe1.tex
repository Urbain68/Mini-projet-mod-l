\chapter*{Annexe 1}
\addcontentsline{toc}{chapter}{Annexe 1}

%changer le format des sections, subsections pour apparaittre sans le num de chapitre
\makeatletter
\renewcommand{\thesection}{\@arabic\c@section}
\makeatother

%recommencer la numérotation des section à "1"
\setcounter{section}{0}

\section*{Expression du critère de Jury à l'équilibre $(x^*, y^*)$}

Comme mentionné à la fin de la section \ref{sec:etude_theorique}, on peut appliquer le critère de Jury à la matrice $J_2$ dans \ref{eq:matrices equilibre}.

Afin de faciliter le calcul, on pose les notations suivantes :
\begin{align*}
    a_1 &= 1 +\alpha - \beta y^* - 2 \gamma x^* = 1 - \alpha + \beta y^* \\
    a_2 &= - \beta x^* \\
    a_3 &= \sigma y^* \\
    a_4 &= 1 - \rho + \sigma x^* + 2 \nu y^*
\end{align*}

De sorte qu'on a l'égalité suivante :
\begin{equation*}
    J_2(x^*, y^*) = 
    \begin{bmatrix}
        a_1 & a_2 \\
        a_3 & a_4 \\
    \end{bmatrix}
\end{equation*}

Alors le polynôme caractéristique de $J_2(x^*, y^*)$ a pour expression:
\begin{align*}
    \chi(X) &= (X - a_1)(X - a_4) - a_2 a_3 \\
            &= X^2 - (a_1 + a_4) X + a_1 a_4 - a_2 a_3
\end{align*}

On peut noter:
\begin{align*}
    b_1 &= - (a_1 + a_4) \\
    b_2 &= a_1 a_4 - a_2 a_3
\end{align*}

De sorte que $\chi(X) = X^2 + b_1 X + b_2$.

Alors les trois conditions du critère de Jury pour les polynômes d'ordre 2 s'écrivent :
\begin{align}
    & b_2 + b_1 + 1 > 0 \\
    & b_2 - b_1 + 1 > 0 \\
    &|b_2| - 1 < 0
\end{align}