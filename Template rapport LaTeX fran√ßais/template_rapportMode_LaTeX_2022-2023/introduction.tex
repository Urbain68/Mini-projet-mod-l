\chapter{Introduction}
\label{ch:Introduction}

\section{Contexte et objectif}
\label{sec:Contexte}

La modélisation des dynamiques de populations a toujours été un sujet d'intérêt majeur afin de comprendre l'évolution de la biodiversité. En effet, la modélisation de l'évolution d'une population permet d'anticiper les espèces en danger et de planifier la protection de la biodiversité.

En particulier, les systèmes proie-prédateurs permettent de modéliser les interactions entre différentes espèces. On peut ainsi prévoir la prolifération d'espèces dangeureuses en modélisant la dynamique de chasse entre des proies et des prédateurs. La formalisation à travers un système d'équations différentielles multiples permet aussi de prendre en compte l'influence de l'environnement.

Le principal défi avec de tels systèmes est l'étude de la stabilité, et donc de la pertinence, des solutions trouvées sur le temps long. En effet, les systèmes proie-prédateur ont tendance à avoir des comportements chaotiques.

\section{Etat de l'art}
\label{sec:Etat}

Historiquement, l'étude de systèmes proie-prédateur est très liées à la modélisation des populations. En particulier, comme le rappelle \cite{BrauerBiology}. En effet, dans un tel système, on considère que la population de proie seule est capable de proliférer. Se pose donc la question du modèle à utiliser afin de représenter la disponibilité des ressources.
Les modèles les plus classiques afin de décrire des sytèmes proie-prédateurs sont rappelés dans \cite{Holl2stoch} sous la forme générale :
\begin{equation}
    \begin{cases}
        \dot{x}(t) = a x(t) - \phi (x(t)) y(t) \\
        \dot{y}(t) = -b y(t) + c \phi (x(t)) y(t)
    \end{cases}
\end{equation}
avec différents choix possibles pour la fonction $\phi$ en fonction de la modélisation choisie.

Après avoir obtenu un système différentiel cohérent, il s'agit ensuite de le transformer en un système à temps discret. En effet, on peut dans une bonne approximation étudier les interactions proie prédateur de manière discréte dans le temps, tous les six mois par exemple.

\section{Contribution}
\label{sec:Contribution}

Ce rapport se propose d'étudier un système proie-prédateur dans lequel on prend en compte le nombre fini de ressource à disposition de chaque espèce. Cela revient à dire que s'il y a plus de proies à ressources constantes, alors les différents membres de cette espèce vont rentrer en compétition les uns avec les autres pour les ressources disponibles.
Concrètement, cela revient à ajouter un facteur d'atténuation de la forme $- x(t)^2$ (resp. $- y(t)^2$) à l'équation régissant la population de proies (resp. de prédateurs).

Afin de mieux mettre en évidence l'influence de ce facteur, l'équation de base choisie est assez simple. Comme on le verra ci-après, le modèle étudié se base sur une variation du modèle proposé traditionnellement par Lotka et Volterra et décrit dans \cite{}. 

\section{Structure du rapport}
\label{sec:Structure}

Le Chapitre \ref{ch:modele_discret} présente le \textit{modèle mathématique à état discret}. Le Chapitre \ref{ch:simu} porte sur l’\textit{analyse des résultats de simulation obtenus en utilisant le modèle présenté au chapitre précédent}. Le Chapitre \ref{chapitre:etat_discret} se focalise sur le développement du \textit{modèle à évènements discrets}. Finalement, les conclusions et perspectives sont présentées.
